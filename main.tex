\documentclass[11pt]{article}

% Font and encoding
\usepackage[utf8]{inputenc}
\usepackage{stix}
\usepackage[T1]{fontenc}

\usepackage[english]{babel}  

% Font manipulation
\usepackage{microtype}

% Figure input and wrapping
\usepackage{graphicx,color}
\usepackage{wrapfig}

% Mathematics
\usepackage{amsmath}

% No indentation
\usepackage{indentfirst}
\setlength{\parindent}{0pt}

% Page geometry
\usepackage{geometry}
\geometry{
	paper=a4paper,
	top=2.5cm, % Top margin
	bottom=2.5cm, % Bottom margin
	left=2cm, % Left margin
	right=2cm, % Right margin
	%showframe,
}

% Hyperlink reference in paper
\usepackage{hyperref}

% Bibliography management
%\usepackage{natbib}
%\bibpunct{[}{]}{,}{a}{,}{;} % open, close, year-author sep, (s:exposant, n:numerical, else:year-author)
\usepackage[round]{natbib}
\setcitestyle{authoryear, open={(},close={)}}

% Notes and comments
\usepackage{todonotes}
\usepackage{easyReview}

% Title author and date
\title{State of the art}
\author{Victor Paredes}
\date{Novembre 2020}

\begin{document}

\maketitle

\tableofcontents

\section{Music}

\subsection{Learning}
\subsubsection{Collaborative learning}
\textit{Collaborative Learning in Music Education: A Review of the Literature} \citep{luce_collaborative_2001}
\begin{itemize}
    \item collaborative learning is defined by three principles :
    \begin{enumerate}
        \item "knowledge is socially constructed as a consensus among the members of a community of knowledgeable peers”
        \item "the authority of knowledge is shared among the members of the community"
        \item "interdependent personal relationships shape a community of knowledgeable peers"
    \end{enumerate}
    \item through collaborative learning, "students would thus become engaged in the exploration of the knowledge and processes involved in the evolution of a music that enlivens and motivates them to participate in music"
    \item in a collaborative learning environment, the responsibility of maintaining the integrity and vitality of music is shared between teachers and students as they form a "community of knowledgeable peer".
\end{itemize}

\textit{Collaborative learning in higher music education} \citep{gaunt_collaborative_2016} -
collection of research articles and reports on collaborative learning practices in higher music education. General observations drawn from the articles :
\begin{itemize}
    \item collaborative learning appears to help the participants reflect on and express their fundamental values as artists and/or pedagogues;
    \item contrary to expectations, individual art forms are not muzzled when engaging in a collaborative learning practice, but rather deepened through its experience.
\end{itemize}

\textit{Collaborative Learning with Interactive Music Systems} \citep{marquez-borbon_collaborative_2020}
\begin{itemize}
    \item four participants were given a newly designed DMI and put in a collaborative learning setup consisting of several workshops over the course of 6 months;
    \item the collaborative process allowed for the definition of learning goals by the team that both oriented and motivated practice;
    \item the group succeeded in developing common set of learning methods and techniques, thus acquiring a similar virtuosity on the instrument, as well as individual styles of playing, bringing complementary to each other styles.
\end{itemize}


\subsubsection{Appropriation}
\textit{The Problem of DMI Adoption and Longevity: Envisioning a NIME Performance Pedagogy} \citep{marquez-borbon_problem_2018}
\begin{itemize}
    \item NIME's ecologically motivated pedagogy is oriented toward the design and the creation of DMIs rather than the development of musical performance;
    \item the authors suggest that the longevity of a DMI should be sought by questioning how well it connects with performance practices rather than how easy it is to adopt it;
    \item this questioning must consider the academic and professional communities that support those practices.
\end{itemize}

\textit{Dimensionality and Appropriation in Digital Musical Instrument Design} \citep{zappi_dimensionality_2014}
\begin{itemize}
    \item 
    \item 
    \item 
\end{itemize}


\subsubsection{instrumental constraint and creativity}
\textit{Dimensionality and Appropriation in Digital MusicalInstrument Design} \citep{zappi_dimensionality_2014}
\begin{itemize}
    \item 
    \item 
    \item 
\end{itemize}



\subsection{Transmission}



\subsection{Design}
\subsubsection{Adaptability}
\textbf{For learning}\\
\textit{P(l)aying Attention: Multi-modal, multi-temporal music control} \citep{gold_playing_2020}
\begin{itemize}
    \item 
    \item 
    \item 
\end{itemize}

\textbf{For personalisation}\\
\textit{Reflections on Eight Years of Instrument Creationwith Machine Learning} \citep{fiebrink_reflections_2020}
\begin{itemize}
    \item 
    \item 
    \item 
\end{itemize}

\textit{Motion-Sound Mapping through Interaction: An Approach to User-Centered Design of Auditory Feedback Using Machine Learning} \citep{francoise_motion-sound_2018}
\begin{itemize}
    \item 
    \item 
    \item 
\end{itemize}

\subsection{Goals and subjectivity}
\textit{Ecological considerations for participatory design of DMIs} \citep{fyans_ecological_2012}
\begin{itemize}
    \item 
    \item 
    \item 
\end{itemize}


\section*{to be classified}
\textit{SoundGuides:Adapting Continuous Auditory Feedback to Users} \citep{francoise_soundguides_2016} 
\begin{itemize}
    \item 
    \item 
    \item 
\end{itemize}

\textit{Exploring different movement sonification strategiesfor rehabilitation in clinical settings} \citep{bevilacqua_exploring_2018}
\begin{itemize}
    \item 
    \item 
    \item 
\end{itemize}

\textit{De-Mo: designing action-sound re-lationships with the mo interfaces} \citep{bevilacqua_-mo_2013}
\begin{itemize}
    \item 
    \item 
    \item 
\end{itemize}

\textit{Sensori-Motor Learning with Movement Sonification:Perspectives from Recent Interdisciplinary Studies} \citep{bevilacqua_sensori-motor_2016}
\begin{itemize}
    \item 
    \item 
    \item 
\end{itemize}

\textit{Modular musical objects towards embodied control of digital music} \citep{rasamimanana_modular_2011}
\begin{itemize}
    \item 
    \item 
    \item 
\end{itemize}


\newpage
\bibliographystyle{dinat}
\bibliography{bibliographie_victor.bib}

\end{document}